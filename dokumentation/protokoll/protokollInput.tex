Ett CAN-meddelande är uppbyggt av ett antal fält med olika funktioner. Det finns bland annat fält för startbitar, identifiering (ID), data, kontrollsummor samt stoppbitar. Av dessa fält är det i huvudsak ID och datafältet som bär den aktuella informationen.
De övriga fälten används av CAN-protokollet för att åstadkomma tillförlitlig kommunikation.

Den inbyggda prioriteringen av meddelanden har noga avvägts vid tilldelning av de 11 bitar som finns tillgängliga i ID-fältet.
Då det är detta fält som avgör prioritringen för meddelanden på bussen så beskriver de i huvudsak meddelandetypen.
De 11 bitarna har fördelats enligt tabell \ref{tab:idbitar}.

\begin{table}[H]
	\centering
	\begin{tabular}{|c|p{0.8\textwidth}|}
		\hline
		Bitar 	& Används till \\ \hline \hline
		0-2		& De 3 första bitarna beskriver meddelandetypen. \\ \hline
		3		& En bit för riktning från/till centralenheten. \\ \hline
		4-10	& 7 bitar:
		Om riktning är till centralenheten är bitarna sändarens ID.
		Om riktningen är från centralenheten beskriver bitarna mottagarens ID. \\ \hline

	\end{tabular}
	\caption{Beskriver hur de 11 bitarna i ID-fältet av CAN-meddelandet används.}
	\label{tab:idbitar}
\end{table}


Datafältets maximala längd är 8 bytes och bildar tillsammans med ID-fältet unika meddelanden. De meddelanden som används beskrivs nedan.


\todo[inline]{Beskrivningarna av meddelandena nedan är mycket övergripande då meddelandena inte är implementerade än.}

\textbf{Ack}
\begin{itemize}
    \item Meddelandetyp 0
    \item Skickas i båda riktningar
\end{itemize}
Bekräftelse för senaste mottagna meddelandet. Detta meddelande har högsta prioritet och skickas som svar på något annat meddelande som inte är av typen ack. \\


\textbf{Skicka larm}
\begin{itemize}
    \item Meddelandetyp 1
    \item Skickas från periferienhet till centralenheten
\end{itemize}
Talar om för centralenheten om någon dörr eller sensor larmar. Detta meddelande har typnummer 1 och skickas endast av periferienheterna. \\


\textbf{Skicka konfiguration}
\begin{itemize}
    \item Meddelandetyp 2
    \item Skickas i båda riktningar
\end{itemize}
Detta meddelande används för att skicka konfigurationer i båda riktningarna.
Centralenheten skickar konfiguration till periferienheten kontinuerligt under drift för att fösäkra sig om att periferienheten har rätt konfiguration. Detta fungerar även som ping, då ett bestämt antal uteblivna ack tolkas som att enheten inte längre finns på nätverket.
I så fall uppmärksammar centralenheten detta.

Periferienheterna skickar sina konfigurationener i andra riktingen vid uppstart.
Dessa ger centralenheten förslag på hur enheterna borde konfigureras initialt. Rörelselarmet skickar även mätvärden i denna typ av meddelande.


\textbf{Tilldelning av ID}
\begin{itemize}
    \item Meddelandetyp 3
    \item Skickas från centralenheten till preferienhet
\end{itemize}
Detta meddelande används då centralenheten tilldelar en periferienhet dess ID. I datafältet skickas ett slumptal som periferienheten använder som identifiering vid tilldelningen av ID. Detta meddelande skickas bara då centralenheten är i konfigurationsläge. Periferienheterna adresseras med bara nollor i de sju bitarna avsedda för ID. \\


\textbf{ID-begäran}
\begin{itemize}
    \item Meddelandetyp 4
    \item Skickas från periferienhet till centralenheten
\end{itemize}
Detta meddelande skickas som en begäran av ID. De sju bitar i ID-fältet som annars ska ange periferienhetens ID är bara nollor i detta meddelande.
