Den inbyggda prioriteringen av meddelanden har noga avvägts vid tilldelning av de 29 bitar som finns tillgängliga i ID-fältet.
Då det är detta fält som avgör prioritringen för meddelanden på bussen så beskriver de i huvudsak meddelandetypen.
De 29 bitarna har fördelats enligt tabell \ref{tab:idbitar}.

\begin{table}[H]
	\centering
	\begin{tabular}{|c|p{0.8\textwidth}|}
		\hline
		Bitar 	& Används till \\ \hline \hline
		28-26		& De 3 första bitarna beskriver meddelandetypen. \\ \hline
		25		& En bit för riktning från/till centralenheten. \\ \hline
		24-18	& 7 bitar:
		Om riktning är till centralenheten är bitarna sändarens ID.
		Om riktningen är från centralenheten beskriver bitarna mottagarens ID. \\ \hline
		17-8 & 10 bitar för session ID, gemensamt ID för sessionen. Används för att förhindra replay attack. \\ \hline
		7-0 & 8 bitar för meddelande nummer. Används bara av meddelanden som bekräftas med ack. \\ \hline

	\end{tabular}
	\caption{Beskriver hur de 29 bitarna i ID-fältet av CAN-meddelandet används.}
	\label{tab:idbitar}
\end{table}


Datafältets maximala längd är 8 bytes och bildar tillsammans med ID-fältet unika meddelanden. De meddelanden som används beskrivs nedan.


\todo[inline]{Beskrivningarna av meddelandena nedan är mycket övergripande då meddelandena inte är implementerade än.}

\textbf{Ack}
\begin{itemize}
    \item Skickas som remote meddelande
    \item Skickas i båda riktningar
\end{itemize}
Bekräftelse för senaste mottagna meddelandet. Man skickar samma header men av remote typ och utan data. \\


\textbf{Skicka larm}
\begin{itemize}
    \item Meddelandetyp 0
    \item Skickas från periferienhet till centralenheten
\end{itemize}
Talar om för centralenheten om någon dörr eller sensor larmar. Detta meddelande skickas endast av periferienheterna. I första byten talar man om vilken typ av sensor som larmar, i andra byten skickas IDt på sensorn\\


\textbf{Skicka konfiguration}
\begin{itemize}
    \item Meddelandetyp 1
    \item Skickas från centralenheten till preferienhet
\end{itemize}
Detta meddelande används för att skicka konfigurationer till periferienheterna.
Centralenheten skickar konfiguration till periferienheten kontinuerligt under drift för att försäkra sig om att periferienheten har rätt konfiguration. Detta fungerar även som ping, då ett bestämt antal uteblivna ack tolkas som att enheten inte längre finns på nätverket.
I så fall uppmärksammar centralenheten detta.
Första databyten anger enhetstyp enligt tabell \ref{tab:enhetstyper}.
\begin{table}
	\centering
	\begin{tabular}{|c|p{0.8\textwidth}|}
		\hline
		Nummer & Används till \\ \hline \hline
		0 & Dörrenhet. \\ \hline
		1 & Rörelseenhet. \\ \hline

	\end{tabular}
	\caption{Nummer enhetstyper.}
	\label{tab:enhetstyper}
\end{table}


\textbf{Tilldelning av ID}
\begin{itemize}
    \item Meddelandetyp 2
    \item Skickas från centralenheten till preferienhet
\end{itemize}
Detta meddelande används då centralenheten tilldelar en periferienhet dess ID. Detta meddelande skickas bara då centralenheten är i konfigurationsläge. I första byten i datafältet skickas ett samma slumptal som skickade i begäran, i andra byten skickas det tilldelade IDt. \\


\textbf{ID-begäran}
\begin{itemize}
    \item Meddelandetyp 3
    \item Skickas från periferienhet till centralenheten
\end{itemize}
Detta meddelande skickas som en begäran av ID. De sju bitar i ID-fältet som annars ska ange periferienhetens ID är bara nollor i detta meddelande. I första databyten skickas ett slumptal som används för idenifiering innan man fått sitt ID.
