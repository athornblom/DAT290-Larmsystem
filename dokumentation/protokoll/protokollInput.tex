Den inbyggda prioriteringen av meddelanden har noga avvägts vid tilldelning av de 29 bitar som finns tillgängliga i ID-fältet.
Detta fält avgör nämligen meddelandets prioritet på bussen.
De 29 bitarna har fördelats enligt tabell \ref{tab:idbitar}. Meddelandetypen anges i de tre första bitarna då den är mest relevant för att avgöra meddelandets prioritet. Detta innebär t.ex. att ett larmmeddelande alltid har högre prioritet än alla andra typer oavsett enhetens ID och andra parametrar i fältet.

\begin{table}[H]
	\centering
	\begin{tabular}{|c|p{0.8\textwidth}|}
		\hline
		Bitar 	& Används till \\ \hline \hline
		28-26		& De 3 första bitarna beskriver meddelandetypen. \\ \hline
		25		& En bit för riktning från/till centralenheten. \\ \hline
		24-18	& Anger mottagarens eller sändarens ID, beroende på meddelandets riktning. (Centralenheten har inget ID.) \\ \hline
		17-8 & 10 bitar för sessions-ID. Detta ID är unikt för varje session mellan två enheter och används för att förhindra replay-attack. \\ \hline
		7-0 & Sekvensnummer. Används bara av meddelanden som bekräftas med ack. \\ \hline

	\end{tabular}
	\caption{Beskriver hur de 29 bitarna i ID-fältet av CAN-meddelandet används.}
	\label{tab:idbitar}
\end{table}


Datafältets maximala längd är 8 bytes. Det bildar tillsammans med ID-fältet unika meddelanden. De meddelanden som används beskrivs nedan.


\todo[inline]{Beskrivningarna av meddelandena nedan är mycket övergripande då meddelandena inte är implementerade än.}

\textbf{Ack}
\begin{itemize}
	\item Samma meddelandetyp som det meddelande som bekräftas
    \item Skickas som remote-meddelande
    \item Skickas i båda riktningar
\end{itemize}
Detta meddelande bekräftar det senast mottagna meddelandet. Det har samma header men är av typen remote och utan data. \\


\textbf{Skicka larm}
\begin{itemize}
    \item Meddelandetyp 0
    \item Skickas från periferienhet till centralenheten
\end{itemize}
Talar om för centralenheten om någon dörr eller sensor larmar. Detta meddelande skickas endast av periferienheterna. Om det är en dörr som larmar skickas en byte med dörrens ID. Om det är en rörelseenhet anger första byten enligt tabell \ref{tab:sensortyper} vilken typ av sensor som larmar. I andra byten skickas sensorns ID.\\


\textbf{Konfiguration}
\begin{itemize}
    \item Meddelandetyp 1
    \item Skickas från centralenheten till preferienhet
\end{itemize}
Detta meddelande används för att skicka konfigurationer till periferienheterna.
Centralenheten skickar konfiguration till periferienheten kontinuerligt under drift för att försäkra sig om att periferienheten har rätt konfiguration. Detta fungerar även som ping, då ett bestämt antal uteblivna ack tolkas som att enheten inte längre finns på nätverket.
I så fall uppmärksammar centralenheten detta.
Första databyten anger enhetstyp enligt tabell \ref{tab:enhetstyper}.


\textbf{Tilldelning av ID}
\begin{itemize}
    \item Meddelandetyp 2
    \item Skickas från centralenheten till preferienhet
\end{itemize}
Detta meddelande används då centralenheten tilldelar en periferienhet dess ID. Det skickas bara då centralenheten är i konfigurationsläge. I de första fyra bytarna av datafältet skickas samma slumptal som periferienheten skickade i begäran. I den femte byten skickas det nya ID:t. \\


\textbf{ID-begäran}
\begin{itemize}
    \item Meddelandetyp 3
    \item Skickas från periferienhet till centralenheten
\end{itemize}
Detta meddelande skickas av en periferienhet innan den har tilldelats ett ID. 
De sju bitarna i ID-fältet som annars ska ange periferienhetens ID är bara nollor i meddelandet.
 I datafältets första byte skickas ett slumptal som används för identifiering 
 innan enheten har fått sitt ID.\\
\begin{table}[htb]
	\centering
	\begin{tabular}{|c|c|}
		\hline
		Nummer & Enhet \\ \hline \hline
		0 & Dörrenhet \\ \hline
		1 & Rörelseenhet \\ \hline

	\end{tabular}
	\caption{Besriver nummer för enhetstyper.}
	\label{tab:enhetstyper}
\end{table}

\begin{table}[htb]
	\centering
	\begin{tabular}{|c|c|}
		\hline
		Nummer & Sensortyp \\ \hline \hline
		0 & Rörelsesensor \\ \hline
		1 & Vibrationssensor \\ \hline

	\end{tabular}
	\caption{Beskriver numrering för sensortyper för rörelseenheter.}
	\label{tab:sensortyper}
\end{table}
