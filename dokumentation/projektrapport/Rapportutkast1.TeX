\documentclass{article}
\usepackage[swedish]{babel}
\usepackage[utf8]{inputenc}
\usepackage[T1]{fontenc}
\usepackage{float}
\usepackage{parskip}
\usepackage{todonotes}

\title{Rapportutkast1}
\author{Grupp 13}
\date{Läsperiod 1, HT-2019}

% I utvärderingen granskas hur väl studenterna i sina rapporter har lyckats visa att de genom de teknikutvärderingar, konstruktionsval, tester och analys som de har gjort har förstått de problem de har ställts inför, samt hur de på basis av denna förståelse och utifrån rådande förutsättningar har lyckats agera för att lösa dessa problem.

% Kriterierna är läsarorienterade vilket innebär att de ska bedöma en utomstående läsaresmöjligheter att förstå rapporten. Rapporten ska först och främst ses som ett dokument vars läsare finns inom organisationen där arbetet sker,men det ska kunna läsas och förstås av personer som inte har jobbat med och därför inte är insatta i projektet, exempelvis en nyanställd som inte känner till projektet och dess bakgrund, en anställd i en annan organisation med liknande verksamhet eller en ingenjörsstudent inom datateknik som har gått en liknande utbildning på en annanhögskola.

% En bakåtsyftande projektrapport: “Rapporten beskriver utvecklingen av…”




\begin{document}

%Titelsida
\maketitle
\pagenumbering{gobble}
\newpage

\tableofcontents
\newpage

\pagenumbering{arabic}


\section{Introduktion}
\subsection{Syfte} % Svarar på fårgan "Varför"?
Rapporten syfte är att beskriva utvecklingen av ett larm- och
 låssystem som i grunden består av ARM-baserade MD407-datorer
  som exekverar kod skriven i programmeringsspråket C.

Systemets uppgift är att öka säkerheten och översikten av
 användarens lokaler, genom att möjliggöra för 
 datarepresentationer av önskade dörrar. Användaren 
 kan sedan använda dessa representationer för att konfiguera 
 individuella dörrar efter unika behov, där exempel på 
 konfigurationsalternativ är uppställningstid innan larm,
  eller upplåsningskod. 

Systemet tillåter även hårdvarumässiga utökningar av 
rörelse- och/eller vibrationssensorer.

Användning av systemet förutsätter ingen speciell 
teknisk expertis.


\subsection{Mål} % Svarar på frågan "Vad"?
x antal dörrar per kort, utrustat med rörelsesensorer, alla korten
 uppkopplade till ett centralsystem.
% I detta avsnitt beskrivs koncist samtliga övergripande tekniska mål med projektet, det vill säga, vad som ska konstrueras. De mål som anges här kommer att styra projektets utveckling. När projektet närmar sig sitt slut och den slutliga projektrapporten lämnas in kommer beställaren att kritiskt analysera hur väl projektet lyckats genom att jämföra planens mål med den tekniska konstruktion som redovisas i projektrapporten.

\subsection{Arbetsmetod} % svarar på frågan "hur har vi gått tillväga i det tekniska utvecklingsarbetet"?
\subsubsection{Subgrupper Erik}
\subsubsection{Kommunikation Gustav}
Under projekets gång har arbetsgruppens medlemmar skött den icke-muntliga 
interna kommunikationen via en gemensam Messenger-chatt. 
I chatten har medlemmarna kunnat ställa projektrelaterade 
frågor till varandra, men det har även fungerat som verktyg
 för medlemmarna att uppdatera varandra om när/var/hur de 
 arbetar samt kommande mötestider.
\subsubsection{Möten och arbetstid Adam}
Projektgruppen har haft ett Möte med handledare varje vecka, där
gruppen har gått igenom vad som gjorts sen senaste mötet och hur
mycket tid man har lagt ner. Samt så har eventuella förändringar
diskuterats på dessa möten men stöd ifrån handledare.

Projektgruppen har även haft två bestämda arbetstillfällen varje
vecka där det flesta medlemmar har närvarat. 

utöver de bestämda arbetspassen har även gruppen suttit och 
jobbat i sina subgrupper på lite mer spontana arbetspass. Under
dessa tillfällen har subgrupperna även kunnat tagit hjälp av 
andra subgrupper då gruppen har suttit i labbsalen.

Varje gruppmedlem har även lagt tid utanför dedikerad arbetstid.  

\subsubsection{Fildelning och versionshantering Gustav - kan behöva hjälp git e svårt}
För att förenkla det parallella programmerandet, och för att samla alla filer och dokument 
på ett och samma ställe, så valde arbetsgruppen att använde sig av versionshanteringstjänsten GIT.


\section{Teknisk beskrivning} % Vad är utgångspunkten för i förkonstruktionsarbetet?
\subsection{Teknisk bakgrund Filip}
\subsection{Systemöversikt Ben}
\subsection{Delsystem }
\subsubsection{Centralenheten}
%\label{subsec:centralenhet}
Centralenheten är den enhet som styr och håller koll på övriga enheter.
 Inställningar görs via USART, varpå de vidarebefordras till rätt
  periferienhet via CAN (se \ref{can}). 

Centralenheten har följande två lägen:

\begin{enumerate}
    \item Standardläge: Enheten väntar på larm-meddelanden från 
    periferienheterna eller användarinput via USART. Dessutom 
    skickar den regelbundet  konfigurationen för anslutna 
    dörrar och sensorer till periferienheterna, som förväntas 
    bekräfta konfigurationen. Om en enhet inte svarar efter 
    ett antal meddelanden går larmet.
    \item Konfigurationsläge: Enheten startar i detta läge.
     Här kan användaren konfigurera anslutna dörr- och 
     rörelseenheter via USART. Användaren kan sätta centralenheten
      i detta läge via USART för att lägga till eller konfigurera
       periferienheter.
\end{enumerate}

För att kunna adressera olika periferienheter måste centralenheten 
tilldela dessa varsitt unikt id. Detta görs i konfigurationsläget
genom att varje enhet som saknar id skickar en förfrågan via CAN.
När centralenheten tar emot en sådan förfrågan från en periferienhet, 
skickar den tillbaka det lägsta lediga id:t och skapar en ny instans
av en struktur som representerar periferienheten och lägger den i en
lista. Strukturen innehåller enhetens id och konfiguration (d.v.s.
tider för varje dörr för dörrenheter respektive avstånd för varje
avståndssensor och antal vibrationssensorer för rörelseenheter). 
Id:t används för att addressera meddelanden till samt identifiera 
meddelanden från enheten. Det är också periferienhetens index i 
centralenhetens lista, vilket innebär att det är oerhört effektivt
att hitta rätt enhet i listan när dess id har lästs i CAN-meddelandet.





\subsubsection{Dörrenheten}
%\label{subsec:Dörrenheten}
Dörrenheten är en fristående enhet som kan köras i lokaldrift
men ska för bäst effekt köras uppkopplad mot centralenheten.

Vid uppstart detekterar och initierar dörrenheten automatiskt alla dörrar som
är stängda och förutsätter att ingen dörr som ska vara aktiv är
öppen. Vid initieringen så ger Dörrenheten varje dörr en hårdkodad
standard konfiguaration som sedan centralenhet kan ändra. 
\subsubsection{Rörelsesensorn}
\subsection{CAN-kommunikation}
\label{can}
Nedan finns dokumentationen av protokollet som tagits fram för 
kommunikationen mellan enheterna. Därefter följer en beskrivning 
av hur kommunikationen har implementerats.
\subsubsection{Protokoll}
\section*{Grundidé}
\label{sec:grundide}

Centralenheten sparar all status och datainformation, denna information kopieras regelbundet till perferienheterna via konfigureringsmeddelanden. Centralenheten har 2 lägen.

\begin{enumerate}
	\item Standard running. Här lyssnar man på larm från periferienheterna. Dessutom skickar man regelbundet  konfigurationen för anslutna dörrar och sensorer. Periferienheten svarar ok, om den inte svarar efter ett antal meddelande larmar man.
	\item Vid uppstart är centralenheten i konfigurationsläge. Nu kan man via USART konfigurera anslutna dörrlarmsenheter samt lörelselarm.
\end{enumerate}

Varje perferienhet förväntas ha en unik fysisk adress som kommer användas vid tilldelning av ID adresser. Likt fysiska MAC adresser och logiska IP adresser i ethernet.

\section*{Bitfördelning i identifieringsdelen av meddelandet}
\label{sec:bitfördelning}

Totalt 11 bitar (Utan att utöka med 18 bitar till)
\begin{description}
	\item{0-2:} Första bitarna är meddelandetyp. 3 bitar dvs 8 meddelandetyper.
	\item{3:} en bit för riktning. Från/till centralenheten. Kan ses som en 4e bit i meddelandetypen.
	\item{4-10:} 7 bitar:
		Om riktning är till centralenheten är bitarna sändarens ID.
		Om riktningen är från centralenheten beskriver bitarna mottagarens ID.
		Innan enheten har en egen ID används bara ettor.
\end{description}


\section*{Meddelandetyper lista}
\label{sec:meddelandetyper}

\begin{table}[H]
	\begin{tabular}{|c|l|p{2.6cm}|p{6cm}|}
		\hline
		N& Riktning & Meddelandenamn & Beskrivning \\ \hline \hline
		0 & p -> c & Ack & Acknowledgement för senaste meddelandet. \\ \hline
		1 & p -> c & Skicka larm & Talar för centralenheten om någon dörr eller sensor larmar. \\ \hline
		3 & Båda & Skicka konfiguration & Skickar konfiguration till periferienheten kontinuerligt. Fungerar även som ping. Perferienheterna skickar sina konfigurationener i andra riktingen vid uppstart. Rörelselarmet har mätvärden i denna typ av meddelande. \\ \hline
		4 & c -> p & Tilldelning av ID & Centralenheten tilldelar en periferienhet dess ID. Enheten identifieras med dess fysiska adress i meddelandet. Detta meddelande skickas bara då centralenheten är i konfigurationsläge (läge  2). \\ \hline
		5 & p -> c & Ny enhet här & Centralenheten tilldelar en periferienhet dess ID. Enheten identifieras med dess fysiska adress i meddelandet. Detta meddelande skickas bara då centralenheten är i konfigurationsläge (läge  2). \\ \hline
	\end{tabular}
	\label{tab:meddelandetyper}
\end{table}

\section*{Detaljinformation angående meddelandetyperna}
\label{sec:detaljinfo}
\textbf{Skicka konfiguration} \\
I datadelen av meddelande för dörrlarm borde det finnas:
\begin{itemize}
	\item Ett par bitar för addressen till dörren i fråga.
	\item En aktiveringsbit per dörr.
	\item En bit per dörr för avaktivering av pågående larm.
	\item Ett par bitar för varje tidsfördröjning.
\end{itemize}

I datadelen av meddelande för rörelsesensorlarm borde det finnas:
\begin{itemize}
	\item En aktiveringsbit per sensor
	\item En bit per sensor för avaktivering av pågående larm.
	\item En bit per enhet för aktivering av automatisk översändning av “Skicka larm och mätvärden” meddelanden. Detta kommer användas vid konfiguration av avståndsgivare.
\end{itemize}

\section*{Todo}
\label{sec:todo}
Borde man ha olika meddelandetyper för “Skicka larm” beroende på vilken enhetstyp det rör sig om?

\subsubsection{Implementation}
% Användarhandledning?

\section{Metoder}
\subsection{Verifikation not done}
\subsection{Programbibliotek STM skriv lite}


\section{Resultat och diskussion} % Kan dela upp dessa om vi vill.
% Detta avsnitt redovisar resultatet av genomförd verifirering av systemet; dels fördelarna, dels för komplett system. Detta avsnitt redovisar även resultatet av det slutgiltiga fysiska testet när hela systemet körs i skarpt läge. Ni ska främst redovisa resultat i form av funktionalitet, men ta ocks upp prestanda aspekter om dessa är viktiga. Diskutea hur väl ni lyckats med er slutprodukt i förhållande till er projektplan.
\section{Slutsats -}
% Detta avsnitt innehåller en sammanfattning av konstruktionen och en diskussion av resultatet. Om detta är möjligt, dra slutsatser av ert projekt och identifiera förbättringsmöjligheter. (Vilka kan vara användbara för en beställare)

\section{Referenser lägg in när det böhövs}
% Referenser enligtIEEE.

% Avslutningsvis så vill vi påminna om att projektets slutrapport, precis som projektplanen, ska vara projektintern, det väl säga att den ska beskriva det tekniska utvecklingsarbetet och bortse från att projektarbetet faktiskt organiserats inom ramen för en kurs.

\section{Bilagor}

\end{document}
