\documentclass{article}
\usepackage[utf8]{inputenc}

\title{Rapportutkast1}
\author{Grupp 13}
\date{Läsperiod 1, HT-2019}

% I utvärderingen granskas hur väl studenterna i sina rapporter har lyckats visa att de genom de teknikutvärderingar, konstruktionsval, tester och analys som de har gjort har förstått de problem de har ställts inför, samt hur de på basis av denna förståelse och utifrån rådande förutsättningar har lyckats agera för att lösa dessa problem.

% Kriterierna är läsarorienterade vilket innebär att de ska bedöma en utomstående läsaresmöjligheter att förstå rapporten. Rapporten ska först och främst ses som ett dokument vars läsare finns inom organisationen där arbetet sker,men det ska kunna läsas och förstås av personer som inte har jobbat med och därför inte är insatta i projektet, exempelvis en nyanställd som inte känner till projektet och dess bakgrund, en anställd i en annan organisation med liknande verksamhet eller en ingenjörsstudent inom datateknik som har gått en liknande utbildning på en annanhögskola.

% En bakåtsyftande projektrapport: “Rapporten beskriver utvecklingen av…”




\begin{document}

\maketitle

%Titelsida


\section{Introduktion}
\subsection{Syfte} % Svarar på fårgan "Varför"?
Projektet syftar till att utveckla ett dörrlarmsystem riktat i huvudsak till mindre företagslokaler. Larmet är tänkt att användas till att
Larmet ska vara lätt att konfiguera för någon utan teknisk expertis.


\subsection{Mål} % Svarar på frågan "Vad"?
x antal dörrar per kort, utrustat med rörelsesensorer, alla korten uppkopplade till ett centralsystem.
% I detta avsnitt beskrivs koncist samtliga övergripande tekniska mål med projektet, det vill säga, vad som ska konstrueras. De mål som anges här kommer att styra projektets utveckling. När projektet närmar sig sitt slut och den slutliga projektrapporten lämnas in kommer beställaren att kritiskt analysera hur väl projektet lyckats genom att jämföra planens mål med den tekniska konstruktion som redovisas i projektrapporten.

\subsection{Arbetsmetod} % svarar på frågan "hur har vi gått tillväga i det tekniska utvecklingsarbetet"?
\subsubsection{Subgrupper Erik}
\subsubsection{Kommunikation Gustav}
\subsubsection{Möten och arbetstid Adam}
\subsubsection{Fildelning och versionshantering Gustav - kan behöva hjälp git e svårt}


\section{Teknisk beskrivning} % Vad är utgångspunkten för i förkonstruktionsarbetet?
\subsection{Teknisk bakgrund Filip}
\subsection{Systemöversikt Ben}
\subsection{Delsystem }
\subsubsection{Centralenheten }
\subsubsection{Dörrenheten}
\subsubsection{Rörelsesensorn}
\subsection{CAN}
% Användarhandledning?

\section{Metoder}
\subsection{Verifikation not done}
\subsection{Programbibliotek STM skriv lite}


\section{Resultat och diskussion} % Kan dela upp dessa om vi vill.
% Detta avsnitt redovisar resultatet av genomförd verifirering av systemet; dels fördelarna, dels för komplett system. Detta avsnitt redovisar även resultatet av det slutgiltiga fysiska testet när hela systemet körs i skarpt läge. Ni ska främst redovisa resultat i form av funktionalitet, men ta ocks upp prestanda aspekter om dessa är viktiga. Diskutea hur väl ni lyckats med er slutprodukt i förhållande till er projektplan.
\section{Slutsats -}
% Detta avsnitt innehåller en sammanfattning av konstruktionen och en diskussion av resultatet. Om detta är möjligt, dra slutsatser av ert projekt och identifiera förbättringsmöjligheter. (Vilka kan vara användbara för en beställare)

\section{Referenser lägg in när det böhövs}
% Referenser enligtIEEE.

% Avslutningsvis så vill vi påminna om att projektets slutrapport, precis som projektplanen, ska vara projektintern, det väl säga att den ska beskriva det tekniska utvecklingsarbetet och bortse från att projektarbetet faktiskt organiserats inom ramen för en kurs.

\section{Bilagor}

\end{document}
