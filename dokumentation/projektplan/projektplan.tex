\documentclass[a4paper]{article}

\usepackage[swedish]{babel}
\usepackage[T1]{fontenc}
\usepackage[utf8]{inputenc}
\usepackage{graphicx}
\usepackage{array}

\title{Projektplan DAT290 \\ \Large Larmsystem grupp 13}
\author{Filip Borg, Gustav Fåhraeus, Josef Karlsson,\\
            Erik Nilsson, Adam Thörnblom, Ben Wooldridge}
\date{2019-09-xx}

\begin{document}
\maketitle
\pagenumbering{gobble}
\newpage

\tableofcontents
\newpage

\pagenumbering{arabic}


\section{SYFTE}
\label{sec:syfte}

Ansvar - Adam

\section{MÅL}
\label{sec:mål}

Målet med detta projekt är att med hjälp av ett antal mikrokontrollrar och periferienheter konstruera ett larmsystem vars huvudsakliga uppgift är att varna när en dörr har varit öppen för länge. Varningen kommer att bestå i att en lysdiod tänds vid aktuell dörr efter en viss tid och att en signal, efter ytterligare en tid, sänds till centralenheten, varpå denna larmar med ljud. Från centralenheten skall användaren kunna ställa in ovan nämnda tidsfördröjningar samt larma och avlarma dörrarna.

\section{BAKGRUND}
\label{sec:bakgrund}

Ett larmsystem av den typ som ska konstrueras använder olika typer av sensorer för att ta reda på om en dörr är öppen och om någon är i närheten. Denna information skickas till centralenheten för att vidta lämpliga åtgärder, alternativt hanteras lokalt vid varje dörrenhet.
Lämpliga åtgärder kan vara att tända en lysdiod, larma med ljud eller på annat sätt göra användaren uppmärksam på vad som har uppmätts.

\subsection{Referenser}
\label{sec:referenser}



\subsection{Tekniska förutsättningar}
\label{sec:tekniskaf}
Larmsystemet kommer att konstrueras av tre enheter som kommunicerar med varandra via en CAN-buss. Varje enhet består av en mikrokontroller (MD-407) och ett antal periferienheter. En av enheterna kommer att vara centralenhet och styra de två andra. Dessa två kommer vara koppplade till ett antal dörrar med hjälp av olika sensorer.



\section{SYSTEMÖVERSIKT}
\label{sec:systemö}

Ansvar - Ben

\section{RESURSPLAN}
\label{sec:resurs}

\begin{itemize}
    \item \textbf{Gruppledare:}
    \\
    Adam Thörnblom - (adam.thornblom@gmail.com)
    
    \item \textbf{Administrativt- och teknisk dokumentationsansvarig:} 
    \\
    Gustav Fåhraeus - (gusfah@student.chalmers.se)

    \item \textbf{Kodstandardansvarig:}
    \\
    Josef Karlsson - (effektgubben@gmail.com)

    \item \textbf{Testansvarig:}
    \\
    Filip Borg - (filipborg97@gmail.com) 

    \item \textbf{Resursansvarig:} 
    \\
    Ben Wooldridge - (benpontus17@gmail.com)

    \item \textbf{Planeringsansvarig:}
    \\
    Erik Nilsson - (ernilss@student.chalmers.se) 
\end{itemize}

Chalmers lokaler, och då främst diverse arbetsrum i EDIT-huset samt Idé-\\läran, kommer användas för gruppmöten och som arbetsyta.

Hårdvaran för projektet kommer att förvaras i grupprum 4211 i EDIT-huset. Hårdvaran som ska finns tillgänglig är:
\begin{itemize}
    \item 3x MD407 kort 
    \item 1x Avståndsmätare (ultraljud), HC-SR04 
    \item 1x Vibrationssensor, "Flying-Fish" SW-18010P 
    \item 1x Keypad 
    \item 1x 7-segmentsdisplay 
    \item 2x 4-polig RJ-11 kabel (används för CAN-bussen) 
    \item 1x RJ-11 förgrening
    \item 2x Tiopolig flatkabel 
    \item 3x USB-kabel 
    \item 1x Kopplingsplatta
\end{itemize}

Kopplingskablar finns i projektrummet, dvs rum 4211 i EDIT-huset. Dörrsensor fås tillgång till genom att prata med handledare Rasmus Edvardsson.

Den nyaste fungerande versionen av mjukvaran som skrivs av gruppmedlemmarna fås tillgång till genom GIT där en gemensam repository för projektet har satts upp.

Arbete som sker utanför Chalmers kommer också bedrivas, både i grupp och enskilt. Detta kommer vara helt mjukvaruorienterat, om frågor skulle uppstå som kräver andra gruppmedlemmars uppmärksamhet så kommer dessa kunna nås via mail, eller via den gemensamma Messenger-gruppen som satts upp.

\section{MILSTOLPAR}
\label{sec:milstolpar}

Ansvar - Filip

\section{AKTIVITETER}
\label{sec:sktiviteter}
Gruppens totala arbetstid på 1400 timmar planeras läggas ner enligt följande tabell.

\textit{Under granskning: Komma på ifall några fler moment bör tilläggas, gå över ifall tiderna är rimliga, bestämma ifall programeringstiden bör delas upp och isf hur, bestämma tid för ev. extrauppgifter. Det finns just nu 35h kvar)}
\begin{table}[htb]
\begin{tabular}{|l|l|}
\hline
\multicolumn{2}{|c|}{\textbf{Aktiviteter}}                                   \\ \hline
\textit{Arbetsmoment}                                     & \textit{Tid (h)} \\ \hline
Genomläsning av uppgift                                   & 10               \\
Initial-genomläsning av dokumentation för periferienheter & 15               \\
Framtagning av \LaTeX \,mallar                     & 5                \\
Projektledning                                            & 10               \\
Projektmöten                                              & 175              \\
Projektplan                                               & 35               \\
Programmering (Totalt) (Bör vi dela upp i områden här?)   & 800              \\
Kod-dokumentation granskning                              & 30               \\
Funktiontestning                                          & 20               \\
Prestandatestning                                         & 10               \\
Dokumentera testning                                      & 10               \\
Granskning \& Finputsning av kod                          & 30               \\
Yttre dokumentation (Slutgiltig systembeskrivning)        & 25               \\
Första utkast slutrapport                                 & 50               \\
Oppositionsrapporter                                      & 20               \\
Andra utkast slutrapport                                  & 20               \\
Förbered demonstration                                    & 30               \\
Demonstration                                             & 10               \\
Renskrivning slutrapport                                  & 30               \\
Kamratuppskattning                                        & 5                \\
Medverkansrapport                                         & 15               \\
Granskning medverkansrapport                              & 10               \\ \hline
\end{tabular}
\end{table}

\section{TIDSPLAN}
\label{sec:tidsplan}

Ansvar - Erik

\section{MÖTESPLAN}
\label{sec:mötesplan}

Ansvar - Adam

\section{KOMMUNIKATIONSPLAN}
\label{sec:komm}

\begin{tabular}{|l|l|l|l|}
\hline
\multicolumn{1}{c}{\bfseries Vad} & \multicolumn{1}{c}{\bfseries När} & \multicolumn{1}{c}{\bfseries Till} & \multicolumn{1}{c}{\bfseries Hur} \\ \hline
Dagordning möte LV1 & 2019-09-02 & c & d \\ \hline
Mötesprotokoll möte LV1 & 2019-09-03 & c & d \\ \hline

Dagordning möte LV2 & 2019-09-11 & c & d \\ \hline
Mötesprotokoll möte LV2 & 2019-09-13 & c & d \\ \hline

Projektplan & 2019-09-15 & c & d \\ \hline

Dagordning möte LV3 & 2019-09-18 & c & d \\ \hline
Mötesprotokoll möte LV3 & 2019-09-20 & c & d \\ \hline

Dagordning möte LV4 & 2019-09-25 & c & d \\ \hline
Mötesprotokoll möte LV4 & 2019-09-27 & c & d \\ \hline

Dagordning möte LV5 & 2019-10-02 & c & d \\ \hline
Mötesprotokoll möte LV5 & 2019-10-04 & c & d \\ \hline

Rapportutkast 1 & 2019-10-06 & c & d \\ \hline

Dagordning möte LV6 & 2019-10-09 & c & d \\ \hline
Oppositionsrapport & 2019-10-10 & c & d \\ \hline
Mötesprotokoll möte LV6 & 2019-10-11 & c & d \\ \hline

Dagordning möte LV7 & 2019-10-16 & c & d \\ \hline
Mötesprotokoll möte LV7 & 2019-10-18 & c & d \\ \hline
Rapportutkast 2 & 2019-10-20 & c & d \\ \hline

Dagordning möte LV8 & 2019-10-23 & c & d \\ \hline
Mötesprotokoll möte LV8 & 2019-10-25 & c & d \\ \hline

Dagordning möte LV9 & 2019-10-30 & c & d \\ \hline
Mötesprotokoll möte LV9 & 2019-11-01 & c & d \\ \hline
Projektrapport & 2019-11-03 & c & d \\ \hline
Kamratuppskattning och medverkansrapport & 2019-11-03 & c & d \\ \hline



\end{tabular}
\\ \\
Kommunikation kommer även ske via en gruppchatt via Messenger-appen, och via logsen på GitHub.
\\
// Lite osäker på om det blir bloated att ha med alla veckomöten, samt om det är något annat som ska vara här.
Tänkte vi kunde snacka om det på mötet.

\section{KVALITETSPLAN}
\label{sec:kval}

Ansvar - Filip

\section{SPELREGLER}
\label{sec:spelregler}

För förhindring av misskötsel i projektgruppen införs ett straffsystem kallat ”Påföljdsfikan”. Påföljdsfikan fungerar så att efter en gruppmedlems misskötsel skall hen nästa sammankomst bjuda på fika för $S_{namn} = S_{namn} + k_{namn}*25$ kr omm $k_{namn}>1$, där $k_{namn}$ och $S_{namn}$ är varje individuella gruppmedlems kakfaktor, som ökas efter varje misskötsel men kan ej minskas utan benådning, respektive kakskuld som ackumuleras fram tills dess att gruppmedlemmen betalar tillbaka den.\\ \par
Punktlista över straff:
\begin{itemize}
	\item 5–10 min sen till möte, $k_{namn} = k_{namn} + 0.5$
	\item 10+ min sen till möte, $k_{namn} = k_{namn} + 1$
	\item Kommer ej till mötet, $k_{namn} = k_{namn} + 1.5$
	\item Ej klar med veckans uppgifter eller slarvigt gjorda, $k_{namn} = k_{namn} + 1$
	\item Ej påbörjat veckans uppgifter eller icke klar och slarvigt gjorda uppgifter, \indent $k_{namn} = k_{namn} + 2$
\end{itemize}
\par
Gruppmedlemmar kan benådas straff omm de har en ursäkt som godkänds av en annan gruppmedlem i förväg eller majoriteten av resterande medlemmar (3/5) godkänner en försenad ursäkt. $S_{namn}$ och $k_{namn}$ kan minskas, ej under 0, vid överlägsenmajoritetröst (4/5) från resterande medlemmar



%För referenser
%\bibliographystyle{IEEEtran}
%\bibliography{referenser}
\end{document}
