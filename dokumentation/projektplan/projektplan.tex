\documentclass[a4paper]{article}

\usepackage[swedish]{babel}
\usepackage[T1]{fontenc}
\usepackage[utf8]{inputenc}
\usepackage{graphicx}
\usepackage{array}

\title{Projektplan DAT290 \\ \Large Larmsystem grupp 13}
\author{Filip Borg, Gustav Fåhraeus, Josef Karlsson,\\
            Erik Nilsson, Adam Thörnblom, Ben Wooldridge}
\date{2019-09-xx}

\begin{document}
\maketitle
\pagenumbering{gobble}
\newpage

\tableofcontents
\newpage

\pagenumbering{arabic}


\section{SYFTE}
\label{sec:syfte}

Ansvar - Adam

\section{MÅL}
\label{sec:mål}

Målet med detta projekt är att med hjälp av ett antal mikrokontrollrar och periferienheter konstruera ett larmsystem vars huvudsakliga uppgift är att varna när en dörr har varit öppen för länge. Varningen kommer att bestå i att en lysdiod tänds vid aktuell dörr efter en viss tid och att en signal, efter ytterligare en tid, sänds till centralenheten, varpå denna larmar med ljud. Från centralenheten skall användaren kunna ställa in ovan nämnda tidsfördröjningar samt larma och avlarma dörrarna.

\section{BAKGRUND}
\label{sec:bakgrund}

Ett larmsystem av den typ som ska konstrueras använder olika typer av sensorer för att ta reda på om en dörr är öppen och om någon är i närheten. Denna information skickas till centralenheten för att vidta lämpliga åtgärder, alternativt hanteras lokalt vid varje dörrenhet.
Lämpliga åtgärder kan vara att tända en lysdiod, larma med ljud eller på annat sätt göra användaren uppmärksam på vad som har uppmätts.

\subsection{Referenser}
\label{sec:referenser}



\subsection{Tekniska förutsättningar}
\label{sec:tekniskaf}
Larmsystemet kommer att konstrueras av tre enheter som kommunicerar med varandra via en CAN-buss. Varje enhet består av en mikrokontroller (MD-407) och ett antal periferienheter. En av enheterna kommer att vara centralenhet och styra de två andra. Dessa två kommer vara koppplade till ett antal dörrar med hjälp av olika sensorer.



\section{SYSTEMÖVERSIKT}
\label{sec:systemö}

Ansvar - Ben

\section{RESURSPLAN}
\label{sec:resurs}

Ansvar - Gustav

\section{MILSTOLPAR}
\label{sec:milstolpar}

Ansvar - Filip

\section{AKTIVITETER}
\label{sec:sktiviteter}

Ansvar - Ben

\section{TIDSPLAN}
\label{sec:tidsplan}

Ansvar - Erik

\section{MÖTESPLAN}
\label{sec:mötesplan}

Ansvar - Adam

\section{KOMMUNIKATIONSPLAN}
\label{sec:komm}

Ansvar - Gustav

\section{KVALITETSPLAN}
\label{sec:kval}

Ansvar - Filip

\section{SPELREGLER}
\label{sec:spelregler}

Ansvar - Erik

%För referenser
%\bibliographystyle{IEEEtran}
%\bibliography{referenser}
\end{document}
