\documentclass[a4paper]{article}

\usepackage[swedish]{babel}
\usepackage[T1]{fontenc}
\usepackage[utf8]{inputenc}
\usepackage{graphicx}
\usepackage{array}

\title{Projektplan DAT290 \\ \Large Larmsystem grupp 13}
\author{Filip Borg, Gustav Fåhraeus, Josef Karlsson,\\
            Erik Nilsson, Adam Thörnblom, Ben Wooldridge}
\date{2019-09-xx}

\begin{document}
\maketitle
\pagenumbering{gobble}
\newpage

\tableofcontents
\newpage

\pagenumbering{arabic}


\section{SYFTE}
\label{sec:syfte}

Ansvar - Adam

\section{MÅL}
\label{sec:mål}

Målet med detta projekt är att med hjälp av ett antal mikrokontrollrar och periferienheter konstruera ett larmsystem vars huvudsakliga uppgift är att varna när en dörr har varit öppen för länge. Varningen kommer att bestå i att en lysdiod tänds vid aktuell dörr efter en viss tid och att en signal, efter ytterligare en tid, sänds till centralenheten, varpå denna larmar med ljud. Från centralenheten skall användaren kunna ställa in ovan nämnda tidsfördröjningar samt larma och avlarma dörrarna.

\section{BAKGRUND}
\label{sec:bakgrund}

Ansvar - Josef

\subsection{Referenser}
\label{sec:referenser}



\subsection{Tekniska förutsättningar}
\label{sec:tekniskaf}



\section{SYSTEMÖVERSIKT}
\label{sec:systemö}

Ansvar - Ben

\section{RESURSPLAN}
\label{sec:resurs}

Ansvar - Gustav

\section{MILSTOLPAR}
\label{sec:milstolpar}

Ansvar - Filip

\section{AKTIVITETER}
\label{sec:sktiviteter}

Ansvar - Ben

\section{TIDSPLAN}
\label{sec:tidsplan}

Ansvar - Erik

\section{MÖTESPLAN}
\label{sec:mötesplan}

Ansvar - Adam

\section{KOMMUNIKATIONSPLAN}
\label{sec:komm}

Ansvar - Gustav

\section{KVALITETSPLAN}
\label{sec:kval}

Ansvar - Filip

\section{SPELREGLER}
\label{sec:spelregler}

Ansvar - Erik

%För referenser
%\bibliographystyle{IEEEtran}
%\bibliography{referenser}
\end{document}
