\documentclass[a4paper]{article}

\usepackage[swedish]{babel}
\usepackage[T1]{fontenc}
\usepackage{parskip}
\usepackage{multicol}
\usepackage[utf8]{inputenc}

\setlength{\columnsep}{1cm}
\begin{document}
\pagenumbering{gobble}


% Vid utförande av test fylls mallen nedan i, vilket resulterar i ett protokoll som sedan sparas som verifikation. Målet med dessa protokoll är att man utan sakkunskap ska kunna skapa sig en överblick av att systemet fungerar som det ska.

\begin{center}
	\LARGE\textbf{Test protokoll}
\end{center}

\begin{multicols}{2}
\section*{Test formulerat av}
\label{sec:formuleratav}
% Namn på den/de som formulerade testet.



\section*{Utfört av}
\label{sec:namndatum}
% Namn på den/de som utförande testet.



\end{multicols}
\section*{Komponent}
\label{sec:komponent}
%Vilken del av systemet testas?



\section*{Testsyfte}
\label{sec:testsyfte}
%Vad ska testet visa? Är testet till följd av tidigare test?



\section*{Hjälpmedel}
\label{sec:hjalpmedel}
%Lista hjälpmedlen som användes i samband med testet.
\begin{itemize}
	\item 
\end{itemize}



\section*{Utförande}
\label{sec:utfarande}
%Hur ska testet utföras?



\section*{Resultat}
\label{sec:resultat}
%Vad är testresultatet? Hittades några buggar?



\section*{Analys}
\label{sec:analys}
%Vad innebär testets resultat? Behövs fler test av komponenten? Behöver andra komponenter testas ytterligare, och i så fall hur?




\end{document}
