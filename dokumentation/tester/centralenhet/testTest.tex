%Beskrivande titel av testet samt en unik label
\subsubsection{test 1}
\label{test1}

\setlength{\columnsep}{1cm}


% Vid utförande av test fylls mallen nedan i, vilket resulterar i ett protokoll som sedan sparas som verifikation. Målet med dessa protokoll är att man utan sakkunskap ska kunna skapa sig en överblick av att systemet fungerar som det ska.


\begin{multicols}{2}
\subsubsection*{Test formulerat av}
% Namn på den/de som formulerade testet.


\subsubsection*{Utfört av}
% Namn på den/de som utförande testet.



\end{multicols}
\subsubsection*{Komponent}
%Vilken del av systemet testas?



\subsubsection*{Testsyfte}
%Vad ska testet visa? Är testet till följd av tidigare test?



\subsubsection*{Hjälpmedel}
%Lista hjälpmedlen som användes i samband med testet.
\begin{itemize}
	\item 
\end{itemize}



\subsubsection*{Utförande}
%Hur ska testet utföras?



\subsubsection*{Resultat}
%Vad är testresultatet? Hittades några buggar?



\subsubsection*{Analys}
%Vad innebär testets resultat? Behövs fler test av komponenten? Behöver andra komponenter testas ytterligare, och i så fall hur?



