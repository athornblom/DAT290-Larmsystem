\clearpage
%Beskrivande titel av testet samt en unik label
\subsection{Test av återspelningsenheten}
\label{test:replayer}

\setlength{\columnsep}{1cm}


% Vid utförande av test fylls mallen nedan i, vilket resulterar i ett protokoll som sedan sparas som verifikation. Målet med dessa protokoll är att man utan sakkunskap ska kunna skapa sig en överblick av att systemet fungerar som det ska.


\begin{multicols}{2}
\subsubsection*{Test formulerat av}
% Namn på den/de som formulerade testet.
Filip

\subsubsection*{Utfört av}
% Namn på den/de som utförande testet.



\end{multicols}
\subsubsection*{Komponent}
%Vilken del av systemet testas?
Återspelningsenheten


\subsubsection*{Testsyfte}
%Vad ska testet visa? Är testet till följd av tidigare test?
Säkerställa att återspelaren klarar av att spela in och återge en kort kommunikationssekvens.
Testet avser både fördröjningar mellan meddelanden och meddelandenas korrekthet.


\subsubsection*{Hjälpmedel}
%Lista hjälpmedlen som användes i samband med testet.
\begin{itemize}
	\item Återspelare
	\item Brusenhet
\end{itemize}



\subsubsection*{Utförande}
%Hur ska testet utföras?
Koppla ihop återspelaren med brusenheten på CAN-bussen och starta sedan båda enheterna.
Se till att brusenheten ger utskrift för skickade meddelanden, se kommandolistan vid uppstart för att hitta rätt kommando.
Starta nu inspelning på återspelningsenheten, konsultera åter igen kommandolistan.
Nu ska en sekvens av meddelanden skickas på ett kontrollerat sätt på bussen.
Använd funktionen för att skicka ett slumpartat meddelande från brusenheten.
Skicka förslagsvis en kaskad av  ca 5 meddelanden följt av ett längre uppehåll på runt 10 sekunder, skicka slutligen ett par avslutande meddelanden.
Därefter avslutas inspelningen på återspelaren.

Kontrollera nu att brusenheten ger utskrift för mottagna meddelanden och starta därefter återspelningen på återspelningsenheten med cyklisk uppspelning.
\todo[inline]{implementera inställning för att toggla mellan cyklisk och icke cyklisk}
Låt meddelandesekvensen återspelas två gånger.
Lägg ingen större vikt vid att kontrollera meddelandenas innheåll det kan göras efteråt, däremot borde fokus läggas på fördröjningarna.
Avlsuta återspelningen och kontrollera meddelandenas korrekthet genom att bräddra tillbaka genom historiken.


\subsubsection*{Resultat}
%Vad är testresultatet? Hittades några buggar?



\subsubsection*{Analys}
%Vad innebär testets resultat? Behövs fler test av komponenten? Behöver andra komponenter testas ytterligare, och i så fall hur?



