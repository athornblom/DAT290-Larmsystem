\clearpage
%Beskrivande titel av testet samt en unik label
\subsection{Prestandatest rörelseenheten}
\label{test:MotionPrestanda}

\setlength{\columnsep}{1cm}


% Vid utförande av test fylls mallen nedan i, vilket resulterar i ett protokoll som sedan sparas som verifikation. Målet med dessa protokoll är att man utan sakkunskap ska kunna skapa sig en överblick av att systemet fungerar som det ska.


\begin{multicols}{2}
\subsubsection*{Test formulerat av}
% Namn på den/de som formulerade testet.
Erik

\subsubsection*{Utfört av}
% Namn på den/de som utförande testet.
Gustav


\end{multicols}
\subsubsection*{Komponent}
%Vilken del av systemet testas?
Rörelseenheten


\subsubsection*{Testsyfte}
%Vad ska testet visa? Är testet till följd av tidigare test?
Säkerställa att ingen märkbar fördröjning sker vid flera inkopplade rörelsesensorer.

\subsubsection*{Hjälpmedel}
%Lista hjälpmedlen som användes i samband med testet.
\begin{itemize}
	\item Avståndsmätare, HC-SR04
	\item CAN-kabel
	\item Kopplingskabel
	\item Kopplingsbräda
\end{itemize}



\subsubsection*{Utförande}
%Hur ska testet utföras?
Koppla först upp en rörelsesensor till rörelseenheten och koppla upp rörelseenheten till centralenheten via CAN. Starta programmen och skicka standard konfiguration på 20 cm larmavsttånd från centralenheten till rörelsesensorn. Testa att sensorn larmar på 20 cm. Starta nu om och koppla upp fyra extra rörelsesensorer. Testa ifall första sensorn som testades fortfarande larmar på 20 cm. Sensorn ska vara okalibrerad för att inte fuska fram samma värde båda gångerna.

\subsubsection*{Resultat}
%Vad är testresultatet? Hittades några buggar?
Sensorn larmade första gången 20.5 cm bort och larmade på samma avstånd efter det kopplats in 4 till rörelsesensorer.


\subsubsection*{Analys}
%Vad innebär testets resultat? Behövs fler test av komponenten? Behöver andra komponenter testas ytterligare, och i så fall hur?
Gick ej att se skillnad på larmavstånd när flera rörelsesensorer, som tar upp större delen av kod exekveringen, är inkopplade. Alltså kan vi dra slutsatsen att rörelseenheten klarar av att exekvera kod för flertalet sensorer samtidigt utan några problem. Även om larmavståndet ändrats med några procent hade detta fixats av kalibrering.


