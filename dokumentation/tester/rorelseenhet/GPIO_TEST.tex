\clearpage
%Beskrivande titel av testet samt en unik label
\subsection{Test av portar samt lokalt och centralt larm för rörelseenheten}
\label{test:PortMotion}

\setlength{\columnsep}{1cm}


% Vid utförande av test fylls mallen nedan i, vilket resulterar i ett protokoll som sedan sparas som verifikation. Målet med dessa protokoll är att man utan sakkunskap ska kunna skapa sig en överblick av att systemet fungerar som det ska.


\begin{multicols}{2}
\subsubsection*{Test formulerat av}
% Namn på den/de som formulerade testet.
Erik

\subsubsection*{Utfört av}
% Namn på den/de som utförande testet.
Adam


\end{multicols}
\subsubsection*{Komponent}
%Vilken del av systemet testas?
Rörelseenheten


\subsubsection*{Testsyfte}
%Vad ska testet visa? Är testet till följd av tidigare test?
Säkerställa att sensorer kan kopplas in på rörelseenhetens alla portar samt testa ifall både lokalt och centralt larm fungerar.

\subsubsection*{Hjälpmedel}
%Lista hjälpmedlen som användes i samband med testet.
\begin{itemize}
	\item Avståndsmätare, HC-SR04
	\item Vibrationssensor, SW-18010P
	\item CAN-kabel
	\item Kopplingskabel
	\item Kopplingsbräda
	\item LED-lampor
\end{itemize}



\subsubsection*{Utförande}
%Hur ska testet utföras?
Testa portarna A, C, D och E genom att koppla en rörelsesensor till port A och en till C samt en vibrationssensor till D och en till E. För rörelsesensorerna kopplas HC-SR04:s trigpin till en pinne på MD407-mikrodatorn som uppfyller 3n (0,3,6.. o.s.v.), HC-SR04:s echopin till pinne (3n + 1) och LED-lampa till pinne (3n + 2). 
För vibrationssensorn kopplas SW-18010P:s DO-pinne till en pinne på MD407-mikrodatorn som uppfyller 2n(0,2,4.. o.s.v.) samt en LED-lampa till pinne (2n + 1). 
se till att rörelseenheten är kopplad  till centralenheten via CAN kabel. Starta centralenheten och håll den i 'startläge', starta nu rörelseenheten och kolla att centralenheten hittar en rörelsesensor på rörelseenheten. Sätt nu centralenheten i 'standardläge' så att den skickar standardkonfigurationer till rörelseenheten, 20 respektive 40 cm centralt och lokalt larmavstånd. Verifiera att LED-lampan till rörelsesensorn börjar lysa när du håller handen ca 30 cm från sensorn. Verifiera sedan att sensorn skickar larm till centralenheten när du håller handen närmare än 20 cm ifrån. Peta på vibrationssensorn och kolla ifall centralenheten får larm.

\subsubsection*{Resultat}
%Vad är testresultatet? Hittades några buggar?
Alla LED-lampor lyste och centralenheten fick larm för både rörelse- och vibratoinssensorn.


\subsubsection*{Analys}
%Vad innebär testets resultat? Behövs fler test av komponenten? Behöver andra komponenter testas ytterligare, och i så fall hur?
Alla rörelseenhetens portar samt lokalt och centralt larm fungerar som de ska.


