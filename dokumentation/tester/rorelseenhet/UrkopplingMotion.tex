\clearpage
%Beskrivande titel av testet samt en unik label
\subsection{Urkopplingslarm rörelseenheten}
\label{test:UrkopplingMotion}

\setlength{\columnsep}{1cm}


% Vid utförande av test fylls mallen nedan i, vilket resulterar i ett protokoll som sedan sparas som verifikation. Målet med dessa protokoll är att man utan sakkunskap ska kunna skapa sig en överblick av att systemet fungerar som det ska.


\begin{multicols}{2}
\subsubsection*{Test formulerat av}
% Namn på den/de som formulerade testet.
Erik

\subsubsection*{Utfört av}
% Namn på den/de som utförande testet.
Filip


\end{multicols}
\subsubsection*{Komponent}
%Vilken del av systemet testas?
Rörelseenheten


\subsubsection*{Testsyfte}
%Vad ska testet visa? Är testet till följd av tidigare test?
Säkerställa att aktiva sensorer som kopplas ur larmar.

\subsubsection*{Hjälpmedel}
%Lista hjälpmedlen som användes i samband med testet.
\begin{itemize}
	\item Avståndsmätare, HC-SR04
	\item Vibrationssensor, SW-18010P
	\item Kopplingskabel
	\item Kopplingsbräda
	\item LED-lampor
\end{itemize}



\subsubsection*{Utförande}
%Hur ska testet utföras?
Dra ut valfri kopplingskabel från en avståndsmätare och en vibrationssensor medan rörelseenheten är underdrift och centralenheten konfigurerat dessa sensorer till aktiva.
Kontrollera att sensorerna larmar till centralenheten vid urkoppling.


\subsubsection*{Resultat}
%Vad är testresultatet? Hittades några buggar?
Larmeddelande skickades för respektive sensor skickades till centralenheten när de kopplades ur.


\subsubsection*{Analys}
%Vad innebär testets resultat? Behövs fler test av komponenten? Behöver andra komponenter testas ytterligare, och i så fall hur?
Går ej att kringå rörelseenhetens larm genom att koppla ur dess sensorer.


