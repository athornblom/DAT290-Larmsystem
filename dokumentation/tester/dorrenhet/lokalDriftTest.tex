\clearpage
%Beskrivande titel av testet samt en unik label
\subsubsection{Test av lokal drift för dörrenhet}
\label{test:localDrift}

\setlength{\columnsep}{1cm}


% Vid utförande av test fylls mallen nedan i, vilket resulterar i ett protokoll som sedan sparas som verifikation. Målet med dessa protokoll är att man utan sakkunskap ska kunna skapa sig en överblick av att systemet fungerar som det ska.


\begin{multicols}{2}
\subsubsection*{Test formulerat av}
% Namn på den/de som formulerade testet.
Adam

\subsubsection*{Utfört av}
% Namn på den/de som utförande testet.
Erik


\end{multicols}
\subsubsection*{Komponent}
%Vilken del av systemet testas?
Dörrenheten


\subsubsection*{Testsyfte}
%Vad ska testet visa? Är testet till följd av tidigare test?
För att säkerställa att dörrenheten körs lokalt ifall ingen central enhet hittas.


\subsubsection*{Hjälpmedel}
%Lista hjälpmedlen som användes i samband med testet.
\begin{itemize}
	\item Ledlampor
	\item Dörrbrytare
	\item Kopplingskapel
	\item kopplingsbräda
\end{itemize}



\subsubsection*{Utförande}
%Hur ska testet utföras?
Starta upp dörrenheten utan att den är kopplad till en centralenhet.
Kontrollera så att efter ca 60 sekunder tänds statuslampan
på dörrenheten. Kolla även att dörrenhetens funktionalitet 
fungerar som den ska. 


\subsubsection*{Resultat}
%Vad är testresultatet? Hittades några buggar?
Lamporna tänds som det ska.



\subsubsection*{Analys}
%Vad innebär testets resultat? Behövs fler test av komponenten? Behöver andra komponenter testas ytterligare, och i så fall hur?
Det fungerar som det ska.


