%Beskrivande titel av testet samt en unik label
\clearpage
\subsection{Test av prestanda för Dörrenheten}
\label{test:prestandaTest}

\setlength{\columnsep}{1cm}


% Vid utförande av test fylls mallen nedan i, vilket resulterar i ett protokoll som sedan sparas som verifikation. Målet med dessa protokoll är att man utan sakkunskap ska kunna skapa sig en överblick av att systemet fungerar som det ska.


\begin{multicols}{2}
\subsubsection*{Test formulerat av}
% Namn på den/de som formulerade testet.
Adam

\subsubsection*{Utfört av}
% Namn på den/de som utförande testet.
Filip


\end{multicols}
\subsubsection*{Komponent}
%Vilken del av systemet testas?
Dörrenheten


\subsubsection*{Testsyfte}
%Vad ska testet visa? Är testet till följd av tidigare test?
Säkerställa att att det inte blir någon märkbar delay när man spammar larm.


\subsubsection*{Hjälpmedel}
%Lista hjälpmedlen som användes i samband med testet.
\begin{itemize}
	\item Ledlampor
	\item Dörrbrytare
	\item Kopplingskabel
	\item Kopplingsbräda
\end{itemize}



\subsubsection*{Utförande}
%Hur ska testet utföras?
Sätt tiden för att larma lokalt till 0 så att det lokala larmet inte har någon delay. Öppna och stäng
sensorn för larm snabbt.


\subsubsection*{Resultat}
%Vad är testresultatet? Hittades några buggar?
Lamporna tänds som de ska och det lokala larmet upplevs reagera snabbt utan fördröjningar. 
Det vill säga att så fort sensorn öppnas tändslampan och släcks när sensorn stängs.



\subsubsection*{Analys}
%Vad innebär testets resultat? Behövs fler test av komponenten? Behöver andra komponenter testas ytterligare, och i så fall hur?
Koden exekveras tillräckligt snabbt utan att några delayer märks för det blotta ögat.


