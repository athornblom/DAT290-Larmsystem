%Beskrivande titel av testet samt en unik label
\subsubsection{Test av pinnar för dörrar}
\label{test:portTest}

\setlength{\columnsep}{1cm}


% Vid utförande av test fylls mallen nedan i, vilket resulterar i ett protokoll som sedan sparas som verifikation. Målet med dessa protokoll är att man utan sakkunskap ska kunna skapa sig en överblick av att systemet fungerar som det ska.


\begin{multicols}{2}
\subsubsection*{Test formulerat av}
% Namn på den/de som formulerade testet.
Adam

\subsubsection*{Utfört av}
% Namn på den/de som utförande testet.
Filip


\end{multicols}
\subsubsection*{Komponent}
%Vilken del av systemet testas?
Dörrenheten


\subsubsection*{Testsyfte}
%Vad ska testet visa? Är testet till följd av tidigare test?
För att säkerställa att alla portar fungerar.


\subsubsection*{Hjälpmedel}
%Lista hjälpmedlen som användes i samband med testet.
\begin{itemize}
	\item Ledlampor
	\item Dörrbrytare
	\item Kopplingskapel
	\item kopplingsbräda
\end{itemize}



\subsubsection*{Utförande}
%Hur ska testet utföras?
Testa portarna A, C, D och E  en åt gången genom att koppla in dörrbrytare på pinnar med jämt index (0,2,4..) och tillhörande ledlampor på pinnarna ovanför (1,3,5..). Kör sedan programkoden för dörrenheten och vänta tills initsieringen är utförd och programet är i lokal drift. Ta nu bort magneten från dörrbrytaren och kontrollera så att tillhörande lampa tänds.


\subsubsection*{Resultat}
%Vad är testresultatet? Hittades några buggar?
Lamporna tänds som det ska.



\subsubsection*{Analys}
%Vad innebär testets resultat? Behövs fler test av komponenten? Behöver andra komponenter testas ytterligare, och i så fall hur?
Det fungerar som det ska.


